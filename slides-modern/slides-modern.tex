%% ============================================================================
%%
%%  Slides til BEGIVENHED
%%
%%  Forfatter: FORNAVN EFTERNAVN
%%
%%  Vigtigt: Kan compiles med Lua(La)TeX (foretrukket) eller pdf(La)TeX!
%%
%%  Bemærkninger:
%%  - Kræver Fira Sans font familien, hvis der compiles med LuaTeX!
%%  - Hvis der skal inkluderes kode, brug da '--shell-escape' (minted-pakken)!
%%
%% ============================================================================

% ~~~~~~~~~~~~~~~~~~~~~~~~~~~~~~~~~~~~~~~~~~~~~~~~~~~~~~~~~~~~~~~~~~~~~~~~~~~~~
% Preamble
% ~~~~~~~~~~~~~~~~~~~~~~~~~~~~~~~~~~~~~~~~~~~~~~~~~~~~~~~~~~~~~~~~~~~~~~~~~~~~~
% ~~~~~~~~~~~~~~~~~~~~~~~~~~~~~~~~~~~~~~
% Documentclass
% ~~~~~~~~~~~~~~~~~~~~~~~~~~~~~~~~~~~~~~
% Kræv etoolbox *før* dokumentklassen indlæses!
% Denne smarte pakke gør det muligt at definere switches/toggles
\RequirePackage{etoolbox}

% Start med at vælge handouts eller almindelige slides!!!
% ( wiki.bath.ac.uk/display/latextricks/Making+handouts+from+your+beamer+presentation )
% ( maths.dur.ac.uk/~sxwc62/blog/2018/latexhandouts/ )
% ( tex.stackexchange.com/questions/292531/ )
\newtoggle{handoutmode}
% \toggletrue{handoutmode}

% Indlæs beamerklassen med passende options
\iftoggle{handoutmode}{%
  \documentclass[handout]{beamer}
  \usepackage{pgfpages}
  \pgfpagesuselayout{4 on 1}[a4paper,border shrink=5mm,landscape]
}{%
  \documentclass{beamer}
}

% Combinationen af etoolbox (toggles) og ifluatex gør os i stand til at checke
% om der compiles med LuaTeX eller 'almindelig' pdfTeX. Hvis det første er
% tilfældet aktiveres Fira Sans som font og andre gode sager.
\usepackage{ifluatex}
\newtoggle{fancyfont}
\ifluatex
\toggletrue{fancyfont}
\else
\togglefalse{fancyfont}
\fi


% ~~~~~~~~~~~~~~~~~~~~~~~~~~~~~~~~~~~~~~
% Beamer og layout
% ~~~~~~~~~~~~~~~~~~~~~~~~~~~~~~~~~~~~~~
% Forskelle på Lua-/XeTeX og pdfTeX. Det drejer sig om pakken til dansk
% orddeling og begreber, samt håndtering af æ, ø og å
\iftoggle{fancyfont}{%
  % luatex
  \usepackage{polyglossia}
  \setdefaultlanguage{danish}
}{%
  % pdftex
  \usepackage[utf8]{inputenc}
  \usepackage[T1]{fontenc}
  \usepackage[danish]{babel}
}

% Indstilling af dansk orddeling
\renewcommand{\danishhyphenmins}{22}

% Brug Metropolis beamer theme (!) og microtype
\usetheme[block=fill]{metropolis}
\usepackage[final]{microtype}
% \usepackage[draft]{microtype}


% ~~~~~~~~~~~~~~~~~~~~~~~~~~~~~~~~~~~~~~
% Math layout
% ~~~~~~~~~~~~~~~~~~~~~~~~~~~~~~~~~~~~~~
% Specificer beamer font-theme (nødvendig for mathspec)
\usefonttheme{professionalfonts}

% Brug også theme-font til matematik (i vægten 'light')
% Bemærk at metropolis også sætter hovedfonten til Fira Sans Light, men det skal
% (åbenbart!) gøres igen, for at det har en effekt.
\iftoggle{fancyfont}{%
  % luatex
  \usepackage{unicode-math}
  \defaultfontfeatures{Ligatures=TeX}
  \setmainfont{Fira Sans Light}
  \setmathfont{TeX Gyre Pagella Math}[Scale=MatchLowercase]  % Fallback
  \setmathfont{Fira Sans Light}[range={up/{num,latin,Latin,greek,Greek},%
    \mathexclam,\mathplus,\pm,\div,\minus,\mathpercent,\mathampersand,%
    \mathquestion,\mathatsign,\increment,\less,\equal,\greater,\ne,\leq,%
    \geq,\matheth,\ell,\partial},%
  Script=Latin,script-features={}, sscript-features={}]
  \setmathfont{Fira Sans Light Italic}[range={it/{latin,Latin,greek,Greek}},%
  Script=Latin, script-features={}, sscript-features={}]
  \setmonofont[Scale=MatchLowercase,BoldFont={Fira Mono Medium}]{Fira Mono}
  % \newfontfamily\docfont[Scale=MatchLowercase]{Linux Libertine O}
  \newfontfamily\docfont[Scale=MatchLowercase]{Arno Pro}
  \newcommand{\whichtex}{\LuaTeX}
}{
  % pdftex
  \usepackage[default]{raleway}
  \usepackage[scaled=.95]{newtxtt}
  \newcommand{\whichtex}{pdf\TeX}
}

% ~~~~~~~~~~~~~~~~~~~~~~~~~~~~~~~~~~~~~~
% Science!
% ~~~~~~~~~~~~~~~~~~~~~~~~~~~~~~~~~~~~~~
% Den sædvanlige matematik
\usepackage{amsmath,mathtools}

% Specificer beamer font-theme (nødvendig for mathspec)
\usefonttheme{professionalfonts}

% Enheder
\usepackage{siunitx}

% Kemi
\usepackage[version=4]{mhchem}


% ~~~~~~~~~~~~~~~~~~~~~~~~~~~~~~~~~~~~~~
% Fremvisning af syntax
% ~~~~~~~~~~~~~~~~~~~~~~~~~~~~~~~~~~~~~~
% Inclusion af kode med fleksibel highlight
% NB: Kræver at Python-modulet 'Pygments' er installeret
% NF: Fejler, hvis ikke der compiles med '--shell-escape' !
% \usepackage{minted}
% \usemintedstyle{friendly}
% \setminted[latex]{autogobble=true,highlightcolor=alertbg}


% ~~~~~~~~~~~~~~~~~~~~~~~~~~~~~~~~~~~~~~
% Farver og lignende
% ~~~~~~~~~~~~~~~~~~~~~~~~~~~~~~~~~~~~~~
% Definition af nogle farver
\usepackage{xcolor}
\definecolor{alertcol}{HTML}{EB811B}   % Taget direkte fra Metropolis-temaet
\definecolor{mDarkTeal}{HTML}{23373b}  % Taget direkte fra Metropolis-temaet
\definecolor{codebg}{RGB}{228,230,231}
\colorlet{alertbg}{alertcol!30}
\definecolor{mintedblue}{RGB}{64,112,161}
\definecolor{mintedgreen}{RGB}{0,112,33}

% Den almindelige 'exampleblock'-farve er lidt svær at læse...
\setbeamercolor{example text}{%
  fg=mintedblue
}

% Baggrunden på 'standout' er normalt inverteret, derfor snyder vi lige lidt.
% Bemærk, at det ikke bare er muligt at skifte metropolis-tema-farve fra lys til
% mørk og tilbage igen; på grund af en bug forbliver teksten efterfølgende hvid,
% på hvid baggrund!
\newcommand{\tipslide}[2]{%
  \begin{frame}
    \begin{center}
      {\Large\bfseries {#1}: \alert{#2}}
    \end{center}
  \end{frame}
}


% ~~~~~~~~~~~~~~~~~~~~~~~~~~~~~~~~~~~~~~
% Blandet
% ~~~~~~~~~~~~~~~~~~~~~~~~~~~~~~~~~~~~~~
% Quotes
\usepackage[danish=guillemets]{csquotes}

% Webpages
\usepackage{url}

% Lav om på lister uden brug af enumitem, for den forstyrrer beamer meget!
%( tex.stackexchange.com/questions/240376 )
\setbeamersize{description width=2.2cm}

% Latinsk tekst
\usepackage{lipsum}

% Flotte tabeller
\usepackage{booktabs}


% ~~~~~~~~~~~~~~~~~~~~~~~~~~~~~~~~~~~~~~
% Lidt makroer
% ~~~~~~~~~~~~~~~~~~~~~~~~~~~~~~~~~~~~~~
% Emulering af smart makro fra memoir
\newcommand{\plainbreak}[1]{\vspace{#1\baselineskip}}

% Tilføj kilder på en smart måde i hhv. ventre og højre nederste hjørne
\usepackage[absolute,overlay]{textpos}
\setbeamerfont{framesource}{size=\tiny}
\newcommand{\lsource}[2][]{%
  \begin{textblock*}{4cm}(0.4cm,8.6cm)
    \begin{beamercolorbox}[ht=0.5cm,left]{framesource}
      \usebeamerfont{framesource}\usebeamercolor[fg]{framesource} {#1} {#2}
    \end{beamercolorbox}
  \end{textblock*}
}
\newcommand{\rsource}[2][]{%
  \begin{textblock*}{4cm}(8.4cm,8.6cm)
    \begin{beamercolorbox}[ht=0.5cm,right]{framesource}
      \usebeamerfont{framesource}\usebeamercolor[fg]{framesource} {#1} {#2}
    \end{beamercolorbox}
  \end{textblock*}
}

% Links til ressourcer
\newcommand\mygithub{\url{https://github.com/jakobmoss/templates}}


% ~~~~~~~~~~~~~~~~~~~~~~~~~~~~~~~~~~~~~~~~~~~~~~~~~~~~~~~~~~~~~~~~~~~~~~~~~~~~~
% Dokument
% ~~~~~~~~~~~~~~~~~~~~~~~~~~~~~~~~~~~~~~~~~~~~~~~~~~~~~~~~~~~~~~~~~~~~~~~~~~~~~
% Titel, forfatter, etc.
\title{Her er en titel}
\subtitle{\small Måske en undertitel}
\author{Fornavn Efternavn}
\date{\today}
\institute{%
  Aarhus Universitet
}


% ~~~~~~~~~~~~~~~~~~~~~~~~~~~~~~~~~~~~~~
% Nu begynder dokumentet
% ~~~~~~~~~~~~~~~~~~~~~~~~~~~~~~~~~~~~~~
\begin{document}


% ~~~~~~~~~~~~~~~~~~~~~~~~~~~~~~~~~~~~~~~~~~~~~~~~~~~~~~~~~~~~~~~~~~~~~~~~~~~~~
% Introduktion
% ~~~~~~~~~~~~~~~~~~~~~~~~~~~~~~~~~~~~~~~~~~~~~~~~~~~~~~~~~~~~~~~~~~~~~~~~~~~~~
% Titelslide!
\begin{frame}[plain]

  \maketitle

\end{frame}


\begin{frame}
  \frametitle{Foredraget vil \dots}

  \textbf{Indeholde}
  \begin{itemize}
  \item En generel gennemgang af \LaTeX{}
  \item Bred introduktion til \alert{mange} emner
  \item Små øvelser undervejs
  \item En pause
  \end{itemize}

  % Her er et eksempel på en pause (= ''animation'')
  \plainbreak1
  \pause

  \textbf{\textsl{Ikke} indeholde}
  \begin{itemize}
  \item Større projekter
  \item Alle detaljer og muligheder
  \item En udtømmende liste af tips og tricks
  \end{itemize}

\end{frame}


% =============================================================================
% DEL 1: LaTeX-101
% =============================================================================
% SLIDE-MED-DEL
\begin{frame}[standout]
  Sådan kan man lave en større inddeling med dette tema!
\end{frame}


% ~~~~~~~~~~~~~~~~~~~~~~~~~~~~~~~~~~~~~~~~~~~~~~~~~~~~~~~~~~~~~~~~~~~~~~~~~~~~~
% Afsnit
% ~~~~~~~~~~~~~~~~~~~~~~~~~~~~~~~~~~~~~~~~~~~~~~~~~~~~~~~~~~~~~~~~~~~~~~~~~~~~~
\section{Et eksempel på et afsnit}

% SLIDE
\begin{frame}[fragile]
  \frametitle{Hvad har jeg lige set?}

  Her i kildekoden er et eksempel på anvendelsen af beamer \emph{overlays} ved
  brug af \verb|<..>|. Bemærk også at slides skal have et \verb|[fragile]| på,
  hvis man vil kunne inkludere kode.

  I venstre side:

  \begin{description}
  \item[Editor]<2-> Programmet du bruger til at skrive \LaTeX{}-koden i
  \item[\alert{Kode}]<2-> Det der kaldes \LaTeX{}. Du skriver dette
  \end{description}

  \plainbreak1

  Og i højre:
  \begin{description}
  \item[Fremviser]<3-> Programmet der viser outputtet
  \item[\alert{Output}]<3-> Resultatet af processen; en PDF-fil
  \end{description}
\end{frame}


% SLIDE med tip
\tipslide{Dagens tip}{Compile meget ofte!}
% Måske dagens vigtigste tip! Hvis man laver fejl i sin kode, brokker LaTeX
% sig; jo oftere man compiler, jo mindre kode kan man nå at lave fejl i ad
% gangen (og som vi skal se, kan LaTeX godt være lidt dårlig til at give
% feedback [indsæt joke om forelæsere/undervisere]).


% SLIDE
\begin{frame}
  \frametitle{Hvorfor bruge \LaTeX{}?}

  På dette slides er overlays igen!

  \textbf{Ulemper}
  \begin{itemize}
  \item<2-> Anderledes end fx Word
  \item<2-> \alert{Indlæringskurven} er stejl
  \item<2-> Tabeller og figurer \textsl{kan} være tunge at danse med
  \end{itemize}


  \medskip

  \textbf{Fordele}
  \begin{itemize}
  \item<3-> Flotte dokumenter! (også med standardopsætningen)
  \item<3-> Adskillelse af \alert{indhold og udseende}
  \item<3-> \alert{Nemt} at håndtere referencer, indholdsfortegnelse, etc.
  \item<3-> Meget aktivt \textsl{community}
  \item<3-> Gratis!
  \end{itemize}
\end{frame}


\begin{frame}
  \frametitle{Tak for opmærksomheden}

  \textbf{Flere templates (især til projekter) kan findes her:}
  \begin{itemize}
  \item \mygithub{}
  \end{itemize}

\end{frame}



% ~~~~~~~~~~~~~~~~~~~~~~~~~~~~~~~~~~~~~~~~~~~~~~~~~~~~~~~~~~~~~~~~~~~~~~~~~~~~~
% Supplemental
% ~~~~~~~~~~~~~~~~~~~~~~~~~~~~~~~~~~~~~~~~~~~~~~~~~~~~~~~~~~~~~~~~~~~~~~~~~~~~~

% Init appendix-mode!
% \appendix

% \begin{frame}[standout]
%   Supplemental
% \end{frame}


\end{document}

% ~~~~~~~~~~~~~~~~~
% Så er vi færdige!
% ~~~~~~~~~~~~~~~~~
