%% ============================================================================
%%
%%  Studieretningsprojekt (SRP)
%%
%%  Forfatter: FORNAVN EFTERNAVN
%%
%%  Preamble
%%
%%  Original preamble: Jakob Lysgaard Rørsted (Mosumgaard)
%% ============================================================================

% ~~~~~~~~~~~~~~~~~~~~~~~~~~~~~~~~~~~~~~~~~~~~~~~~~~~~~~~~~~~~~~~~~~~~~~~~~~~~~
% Core
% ~~~~~~~~~~~~~~~~~~~~~~~~~~~~~~~~~~~~~~~~~~~~~~~~~~~~~~~~~~~~~~~~~~~~~~~~~~~~~
% Essentials
\usepackage[utf8]{inputenc}
\usepackage[T1]{fontenc}

% Microtype -- Subliminal refinements towards typographical perfection
\usepackage{microtype}
\microtypesetup{final}

% Various tools needed in the preamble and by some packages
\usepackage{etoolbox}

% Inclusion of code
% --> Needs to be loaded this early to avoid problems with some font packages
% NOTE: This packages requires the Python-module 'Pygments' to be installed
% NOTE: Fails unless this file is compiled with '--shell-escape' !
% \usepackage{minted}
% \usemintedstyle{friendly}

% Language (has to be loaded before the fonts..?)
\usepackage[danish]{babel}
\renewcommand\danishhyphenmins{22}  % For Danish "hv-ord".

% DANISH STUFF
% Navn på indholdsfortegnelsen
\addto\captionsdanish{\renewcommand{\contentsname}{Indholdsfortegnelse}}

% Fonts: Linux Libertine (http://www.tug.dk/FontCatalogue/linuxlibertine/)
\usepackage{libertine}            % Linux Libertine as text font
\usepackage{libertinust1math}     % Math support for Linux Libertine
\usepackage[scaled=.95]{newtxtt}  % Pretty teletype in correct size
\linespread{1.1}                  % Due to a strict character-per-page limit


% ~~~~~~~~~~~~~~~~~~~~~~~~~~~~~~~~~~~~~~~~~~~~~~~~~~~~~~~~~~~~~~~~~~~~~~~~~~~~~
% Page 'n' stuff
% ~~~~~~~~~~~~~~~~~~~~~~~~~~~~~~~~~~~~~~~~~~~~~~~~~~~~~~~~~~~~~~~~~~~~~~~~~~~~~
% Page set-up (margins
\setlrmarginsandblock{4.0cm}{*}{1}
\setulmarginsandblock{*}{4.7cm}{0.8}

% Some memoir tricks
\setlength{\topskip}{1.6\topskip}
\checkandfixthelayout
\sloppybottom
\strictpagechecktrue


% ~~~~~~~~~~~~~~~~~~~~~~~~~~~~~~~~~~~~~~~~~~~~~~~~~~~~~~~~~~~~~~~~~~~~~~~~~~~~~
% Page and chapter styles
% ~~~~~~~~~~~~~~~~~~~~~~~~~~~~~~~~~~~~~~~~~~~~~~~~~~~~~~~~~~~~~~~~~~~~~~~~~~~~~

% NOTE: Colors added everywhere, to make them easy to change!
\usepackage{xcolor}

% Fonts
\newcommand{\foliofont}{\color{black}\sffamily}
\newcommand{\headerfont}{\color{black}\sffamily}

% Make new pagestyle
\makepagestyle{jlrsrp}
\makeoddhead {jlrsrp}{\headerfont\theauthor} {} {\headerfont\projecttitle}
\makeoddfoot {jlrsrp}{}{\foliofont\thepage}{}

% Black magic happens below! (define pagestyle)
\makeatletter
\makepsmarks {jlrsrp}{
  % Syntax: \createmark{<division type}{left|right|both marks}{shownumber|nonumber}{prefix}{postfix}
  \createmark{chapter}    {both}  {shownumber} {\@chapapp\ } {\ $\cdot$\ }
  \createmark{section}    {right} {shownumber} {}            { \ }
%  \createmark{subsection} {right} {nonumber}   {}            {}
  \createplainmark{toc}   {both}  {\contentsname}
  \createplainmark{lof}   {both}  {\listfigurename}
  \createplainmark{lot}   {both}  {\listtablename}
  \createplainmark{bib}   {both}  {\bibname}
  \createplainmark{index} {both}  {\indexname}
}
\makeatother
\nouppercaseheads

% Sections with sans-serif
\setsecheadstyle{\Large\bfseries\sffamily\raggedright}
\setsubsecheadstyle{\large\bfseries\sffamily\raggedright}

% Fix of the pagestyle of the chapter-pages
\copypagestyle{chapter}{empty}
\makeoddfoot{chapter}{}{\foliofont\thepage}{}
\makeevenfoot{chapter}{}{\foliofont\thepage}{}

% Chapterstyle (modified from Rasmus Villemoes thesis)
\usepackage{graphicx}
\makechapterstyle{jrmbsc}{% requires graphicx package
  \chapterstyle{default}
  \renewcommand*{\chapnamefont}{%
    \normalfont\LARGE\color{black}\scshape\raggedleft}
  \renewcommand*{\chaptitlefont}{%
    \normalfont\Huge\color{black}\sffamily\raggedleft}
  \renewcommand*{\chapternamenum}{}
  \renewcommand*{\printchapternum}{%
    \makebox[0pt][l]{\hspace{0.4em}
      \resizebox{!}{5ex}{%
        \normalfont\Large\color{black}\thechapter}
    }%
  }%
  \renewcommand*{\afterchapternum}{%
    \par\hspace{1.5cm}\color{black}\hrule\vskip\midchapskip}}

% Use the new pagestyle and chapterstyle
\pagestyle{jlrsrp}
\chapterstyle{jrmbsc}

% Turn on numbering of subsections
\setsecnumdepth{subsection}


% ~~~~~~~~~~~~~~~~~~~~~~~~~~~~~~~~~~~~~~~~~~~~~~~~~~~~~~~~~~~~~~~~~~~~~~~~~~~~~
% Floats, captions and footnotes
% ~~~~~~~~~~~~~~~~~~~~~~~~~~~~~~~~~~~~~~~~~~~~~~~~~~~~~~~~~~~~~~~~~~~~~~~~~~~~~
% Include graphics
% \usepackage{graphicx}  % ALREADY IMPORTED

% Subfloats
\newsubfloat{figure}
\subcaptionstyle{\raggedright}

% Use sans-serif for captions (alternative layout: Change width (see below))
\captionnamefont{\sffamily\scshape}
\captiontitlefont{\sffamily\small}

% Width of caption --> Use sf-font instead
% \captionwidth{.8\linewidth}
% \changecaptionwidth

% Trick to automatically end captions with a period
\captiontitlefinal{.}

% Styling of the footnotes (memoir tricks)
\setlength{\footmarkwidth}{-1sp}
\setlength{\footmarksep}{0em}
\footmarkstyle{#1: }

% Cool tables with footnotes (using the same style as define just above)
\usepackage[online]{threeparttable}
\appto\TPTnoteSettings{\footnotesize}


% ~~~~~~~~~~~~~~~~~~~~~~~~~~~~~~~~~~~~~~~~~~~~~~~~~~~~~~~~~~~~~~~~~~~~~~~~~~~~~
% Science
% ~~~~~~~~~~~~~~~~~~~~~~~~~~~~~~~~~~~~~~~~~~~~~~~~~~~~~~~~~~~~~~~~~~~~~~~~~~~~~
% Basic math (might already be loaded by the math font package)
\usepackage{amsmath}

% Bad-ass math!
\usepackage{mathtools}
\mathtoolsset{showonlyrefs=false,showmanualtags}

% Units
\usepackage{siunitx}
\sisetup{separate-uncertainty=true}

% Declaration of some nice units
\DeclareSIUnit\au{AU}
\DeclareSIUnit\year{yr}
\DeclareSIUnit\erg{erg}
\DeclareSIUnit\msun{M_{\odot}}
\DeclareSIUnit\lsun{L_{\odot}}

% Computer-related units
\DeclareSIUnit\byte{B}

% Delimeters
\DeclarePairedDelimiter\abs{\lvert}{\rvert}
\DeclarePairedDelimiter\norm{\langle}{\rangle}


% ~~~~~~~~~~~~~~~~~~~~~~~~~~~~~~~~~~~~~~~~~~~~~~~~~~~~~~~~~~~~~~~~~~~~~~~~~~~~~
% Stuff
% ~~~~~~~~~~~~~~~~~~~~~~~~~~~~~~~~~~~~~~~~~~~~~~~~~~~~~~~~~~~~~~~~~~~~~~~~~~~~~
% Spacing in macros
\usepackage{xspace}

% Debugging
\usepackage{lipsum}
\usepackage[margin,draft]{fixme}
\fxusetheme{color}
% \fxnote (grøn), \fxerror (gul), \fxwarning (orange), \fxfatal (rød)

% Nice itemizations
\usepackage{enumitem}


% ~~~~~~~~~~~~~~~~~~~~~~~~~~~~~~~~~~~~~~~~~~~~~~~~~~~~~~~~~~~~~~~~~~~~~~~~~~~~~
% References
% ~~~~~~~~~~~~~~~~~~~~~~~~~~~~~~~~~~~~~~~~~~~~~~~~~~~~~~~~~~~~~~~~~~~~~~~~~~~~~
% Smart quotations
\usepackage{csquotes}

% URL's
\usepackage{url}

% Bibliography
\usepackage[backend=biber,
  style=numeric,
  uniquelist=false,
  uniquename=false,
  doi=false,
  url=false,
  isbn=false,
  eprint=false,
  hyperref=true]{biblatex}

% Actually apply the citation order (because the bibliography is sorted differently)
\assignrefcontextentries[]{*}

% Load the file with bibliographic information
\addbibresource{bibliography.bib}

% Space in bibliography
\setlength\bibitemsep{1.1\itemsep}

% DANISH STUFF: Et al. på dansk
\DefineBibliographyStrings{danish}{%
  andothers = {et\addabbrvspace al\adddot}
}

% DANISH STUFF: Title of bibliography and entry in toc
\defbibheading{memoirbib}[Litteraturliste]{%
  \chapter*{#1} \addcontentsline{toc}{chapter}{#1}}

% Referencing packages (needs to be loaded in this order!)
% --> For references, use: \cref{}  or \Cref{} !
\usepackage{refcount}
\usepackage{varioref}
\usepackage[
  unicode=true,
  pdftitle={\projecttitle},
  pdfauthor={\theauthor},  % Change this name!
  pdfkeywords={},
  bookmarksopen=true,
  pdfdisplaydoctitle=true,
  hypertexnames=false]{hyperref}
\usepackage{cleveref}


% Hyperref setup (from Rasmus Villemoes)
\makeatletter
\@ifpackageloaded{hyperref}{
  \hypersetup{colorlinks=false, pdfborder=0 0 0}
  \addto\extrasenglish{ % What does this do???
    \renewcommand\subsectionautorefname{Subsection}%
    \renewcommand\sectionautorefname{Section}%
    \renewcommand\chapterautorefname{Chapter}%
    \renewcommand\equationautorefname{equation}%
  }
}{}
\makeatother


% ~~~~~~~~~~~~~~~~~~~~~~~~~~~~~~~~~~~~~~~~~~~~~~~~~~~~~~~~~~~~~~~~~~~~~~~~~~~~~
% Macros
% ~~~~~~~~~~~~~~~~~~~~~~~~~~~~~~~~~~~~~~~~~~~~~~~~~~~~~~~~~~~~~~~~~~~~~~~~~~~~~
% Nice spacing in macros
\usepackage{xspace}

% Small division
\newcommand\starbreak{\medskip\fancybreak{$*$\qquad$*$\qquad$*$}\medskip}

% Seperation in equation
\newcommand\eqsep{\ensuremath{\quad , \quad}}

% Generic names
\newcommand\numpy{NumPy\xspace}
\newcommand\scipy{SciPy\xspace}
\newcommand\kepler{\textsl{Kepler}\xspace}

% Diff
\newcommand\diff[2]{\frac{\text{d}#1}{\text{d}#2}}
\newcommand\pdiff[2]{\frac{\partial #1}{\partial #2}}
\newcommand\idiff[2]{\ensuremath{\text{d}#1/\text{d}#2}}

% More lazy stuff
\newcommand\parname[2]{\left(#1\right)_{\textup{#2}}}

% Vectors (we are doing physics!)
\renewcommand{\vec}[1]{\ensuremath{\boldsymbol{#1}}\xspace}

% Nice subscript
\newcommand\var[2]{\ensuremath{#1_{\textup{#2}}\,}\xspace}

% Specific abbreviations and names
\newcommand\eos{EoS\xspace}
\newcommand\gar{\textsc{garstec}\xspace}
\newcommand\Gar{\textsc{Garstec}\xspace}

% Specific things to be repeated often in the text
\newcommand\teff{\ensuremath{T_{\textup{eff}}}\xspace}
\newcommand\logg{\ensuremath{\log g}\xspace}